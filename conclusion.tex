
We have presented a collection of hierarchical cyber-resiliency assurance patterns, which are bundled with our BriefCASE framework and instantiated with a specific system under development.  Automated instantiation and evaluation of these patterns provides us with confidence that we have adequately analyzed the system for cyber vulnerabilities and addressed the corresponding cyber requirements in the system design \textit{and} implementation.
Although these patterns mainly correspond to the automated BriefCASE features that support the CASE workflow outlined in Fig.~\ref{fig:workflow}, BriefCASE is not required to use them; with minor alteration they can be applied to any tool chain that supports a similar workflow.

%\todo{Discuss evaluation somehow}

%\todo{Any way to discuss pattern completeness?}

%\todo{some discussion whether the assurance cases generated are sufficient to be submitted to the authorities or whether there is additional manual rework required}

We have demonstrated the utility of our tools and methods on several real-world use cases that were subjected to red-team adversarial evaluation\footnote{On the DARPA CASE program, BriefCASE was applied to a section of CH-47 mission control software, as well as an AFRL UxAS application. Additional material including videos and code can be found on our BriefCASE project website at https://loonwerks.com/projects/case.html.}.  Nonetheless, it will not often be the case that an entire high-assurance system can be developed in this fashion.  Not all cyber vulnerabilities can be mitigated by an automated model transformation.  Not all component implementations can be synthesized in a provably correct manner.  And not all evidential development artifacts can be automatically evaluated.  But, for many systems, we are confident that the technologies described in this paper can be usefully employed to improve cyber resilience, even when some elements of the system, e.g. legacy components, resist rigorous analysis and/or automated synthesis.

Although the assurance patterns described herein provide confidence that (a) the CASE workflow was properly followed for a specific system development configuration and (b) the resulting deployable system is acceptably cyber-resilient, additional patterns are still necessary to support typical development efforts we see in practice today.  
%
Our cyber-resiliency patterns are structured hierarchically, enabling straightforward insertion of additional pattern fragments corresponding to new cyber vulnerability mitigations, processes, and workflows. 
We anticipate working towards supporting these patterns in future research projects, with contributions encouraged from the wider security assurance community.