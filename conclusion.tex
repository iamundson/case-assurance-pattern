
In this paper, we have presented a collection of hierarchical cyber-resiliency assurance patterns, which are bundled with our BriefCASE framework and instantiated with a specific system under development.  Automated instantiation and evaluation of the patterns provides us with confidence that we have adequately analyzed the system for cyber vulnerabilities and addressed the corresponding cyber requirements in the system design \textit{and} implementation.
Although these patterns mainly correspond to the automated BriefCASE features that support the CASE workflow outlined in Fig.~\ref{fig:workflow}, BriefCASE is not necessarily required to use them; with minor alteration they can be applied to any tool chain that adheres to the workflow.

Unfortunately, it will rarely be the case that an entire high-assurance system can be developed in this fashion.  Not all cyber requirements can be mitigated by an automated model transformation.  Not all component implementations can be synthesized in a provably correct manner.  And not all evidential development artifacts can be automatically evaluated.
%
Although these assurance patterns provide confidence that (a) the CASE workflow was properly followed for a specific system development configuration and (b) the resulting deployable system is acceptably cyber-resilient, additional patterns are still necessary to support typical development efforts we see in practice today.  
%
Our cyber-resiliency patterns are structured hierarchically, enabling straight forward insertion of additional pattern fragments corresponding to new cyber vulnerability mitigations, processes, and workflows. 
We anticipate working towards filling these gaps on future research projects as well as with contributions from the wider security assurance community.