\documentclass[runningheads]{llncs}

\begin{document}

% Please feel free to suggest a better title
\title{An Assurance Pattern for Cyber-Resilient Systems Engineering}

\author{
	Isaac Amundson\inst{1} 
	\and Darren Cofer\inst{1} 
	\and David Hardin\inst{1} 
	\and John Hatcliff\inst{2}}

\institute{
	Applied Research and Technology, Collins Aerospace, USA \\
	\email{\{isaac.amundson,darren.cofer,david.hardin\}@collins.com}\\
	\and Kansas State University, USA \\
	\email{hatcliff@ksu.edu}
}


\maketitle

\begin{abstract}

\end{abstract}

% Include short CASE / BriefCASE overview
\section{Introduction}
\label{sec:introduction}

% DARPA CASE

% REPHRASE TO AVOID COPYRIGHT CONCERNS

In recent years, aerospace stakeholders have realized that avionics systems are subject to possible cyber-attacks just like other cyber-physical systems. Thus, in addition to being fault tolerant, safety-critical avionics systems must also be cyber-resilient. Cyber-resiliency means that the system is tolerant to cyber attacks just as safety-critical systems are tolerant to random faults: they recover and continue to execute their mission function, or safely shut down, as requirements dictate. Unfortunately, systems engineers are currently given few development tools to help answer even basic questions about potential vulnerabilities and mitigations, and instead rely on process-oriented checklists and guidelines. Cyber vulnerabilities are often discovered during penetration testing late in the development process; or worse yet, they may be discovered only after the product has been fielded, necessitating extremely expensive and time-consuming remediation. This is not a sustainable development model. 

The DARPA Cyber Assured Systems Engineering (CASE) program is targeted at developing tools for design, analysis, and verification that enable systems engineers to \textit{design-in} cyber-resiliency for complex cyber-physical systems. 
%
% BRIEFCASE TOOLS AND FEATURES
Our team developed BriefCASE~\cite{case-at-scale}, a tool chain for developing and assuring cyber-resilient embedded systems according to the workflow depicted in Figure~\ref{fig:workflow}. BriefCASE provides a development environment for modeling system architectures in AADL~\cite{feiler-aadl}, analyzing the models for cyber-vulnerabilities, mitigating those vulnerabilities by applying automated model transformations, formally verifying security properties in the model, generating high-assurance component code from model specifications, building the system to a secure kernel target, and finally, generating a system cybersecurity assurance case.  

\begin{figure}[h] 
	\centering 
	\includegraphics[width=\textwidth]{figs/workflow.png}
	\caption{BriefCASE workflow.}
	\label{fig:workflow} 
\end{figure}

% ASSURANCE
For the development of high-assurance commercial products, there are multiple stakeholders that must be convinced of a product's dependability prior to market-release or deployment.  First and foremost, the developing organization must be confident in the product's correctness.  Next, the certification or accreditation authority must be convinced.  Finally the customer, end user (and often other stakeholders in between) will need to be convinced the product will behave as intended.  Assurance is the process by which an organization compiles a comprehensive argument to demonstrate a product's dependability~\cite{???}.  In order to be effective, the argument must be well-formed, complete, and substantiated with evidence.  Assembling such an argument is not a straight-forward task.  For the highest-regulated environments, it is argued that construction of these arguments should never be left to automation~\cite{???-Holloway}.  In other situations, arguments can be constructed based on assurance templates, or \textit{patterns}, in which generic arguments are defined (ideally arrived at through consensus by a body of experts), then instantiated with a concrete system instance.  This is the approach taken by BriefCASE, which includes a cybersecurity assurance pattern that is instantiated with the system under development, and incorporates evidence produced by artifacts generated by the framework workflow.

% RELATED WORK (Assurance patterns, assurance generation, DMILS, DESTION, what's new here)


In previous work~\cite{resolute-destion}, we described assurance patterns corresponding to specific BriefCASE mitigations.  However, those patterns primarily focused on assuring aspects of security requirements were satisfied in the model by examining the model structure (for example, demonstrating that an inserted filter component could not be bypassed).  
%
% CONTRIBUTION AND PAPER OUTLINE
In this paper we expand on that work by presenting a comprehensive CASE assurance pattern covering the generation and ingestion of cyber requirements, requirement satisfaction in the model and requirement satisfaction in the realization of the model.  

Section~\ref{sec:assurance-pattern} presents the assurance pattern with respect to the CASE workflow. Specifically, we focus on arguments for security requirement correctness, model correctness, and implementation correctness.  In Section~\ref{sec:evaluation} we describe how the CASE assurance pattern is instantiated and evaluated by the BriefCASE framework.  We provide concluding remarks and discuss future directions in Section~\ref{sec:conclusion}.

% Include DMILS - compare/contrast
% or integrate this section with Intro if short on space
\section{Related Work}
\label{sec:related-work}
In previous work~\cite{resolute-destion}, we described assurance patterns corresponding to specific BriefCASE cyber-vulnerability mitigations.  However, we only provided a high-level summary of those patterns, and they primarily focused on assuring that security requirements were satisfied in the design by examining the model structure (for example, demonstrating that an inserted filter component could not be bypassed).
%However, those patterns primarily focused on assuring that aspects of security requirements were satisfied in the design by examining the model structure (for example, demonstrating that an inserted filter component could not be bypassed).  
%Moreover,~\cite{resolute-destion} is a short-format paper that only gives a very high-level summary of the selected patterns.
%
% CONTRIBUTION AND PAPER OUTLINE
In this paper we significantly expand on that work by presenting a more comprehensive collection of hierarchical CASE assurance patterns covering the generation and ingestion of cyber requirements, requirement satisfaction in the model, and requirement satisfaction in the realization of the model.  In addition, we describe the mechanisms by which evidence generated as part of the BriefCASE workflow is incorporated into an instantiated assurance case.

The high-level structure of our CASE assurance patterns is inspired in part by the D-MILS argument pattern~\cite{dmils}, in which system dependability properties are assured via modules arguing component, compositional, and implementation correctness.  Similar to BriefCASE, the authors demonstrate how to instantiate the D-MILS pattern from an AADL system model; however, they accomplish this via specification of an additional \textit{weaving model}, which is not necessary in BriefCASE due to the tight coupling between the modeling environment and assurance tool.

The VERDICT~\cite{verdict} framework was also developed on the CASE program and has some similarities with BriefCASE.  Although VERDICT does generate and evaluate assurance arguments based on analyses performed as part of the tool workflow, the assurance arguments are only shallow fragments of a comprehensive cybersecurity case, and focus primarily on whether applicable Common Attack Pattern Enumerations and Classifications (CAPEC)~\cite{capec} have been addressed in the design.  In contrast, our CASE assurance patterns consider vulnerability mitigations in the system design \textit{and} the implementation, and include arguments for several other aspects of cyber-resiliency assurance as well.


The Architecture-driven Multi-concern and Seamless Assurance and Certification of Cyber-Physical Systems (AMASS)~\cite{AMASS} framework incorporates a workflow similar to BriefCASE, but has a scope that encompasses assurance for all types of dependability properties, and a more ambitious goal of driving down certification costs for high-assurance systems development. AMASS supports tools and processes for system design, verification and validation activities and assurance generation, among others.  It is extensible and has a growing ecosystem supported by a European consortium of researchers and practitioners.  In contrast, BriefCASE is targeted at the development of provably cyber-resilient embedded systems, and although it does support analysis and assurance of other classes of dependability properties, the assurance patterns presented in this paper focus strictly on providing confidence of the security of the system under development.



%\todo{Describe other cybersecurity assurance patterns (NIST, etc)}


% Do we present this in graphical GSN, or Resolute, or both?
% Start high-level, go as low-level as space allows
% Don't go into too much detail of model argument, instead refer to Destion paper
% Spend more time on implementation argument?
\section{Assurance Pattern}
\label{sec:assurance-pattern}
% Overview of assurance argument structure
The high-level structure of our CASE assurance pattern is inspired by the D-MILS argument pattern~\cite{dmils}, in which system dependability properties are assured via modules arguing component, compositional, and implementation correctness.  
Although verifying functional, safety, and other dependability properties is necessary for a comprehensive system assurance case, the CASE pattern presented here only addresses cyber-resiliency.  The intention is for the resulting assurance argument to be integrated into a full system assurance case, if necessary.

The high-level CASE argument structure is depicted in Figure~\ref{fig:top-level}, with the top-level goal being ``The system is acceptably cyber-resilient".  
This goal is then substantiated by arguments that cyber-resiliency requirements have been appropriately identified and then satisfied, both in the system model and the \textit{realization} of the system model as a built, deployable system.

\begin{figure}[h] 
	\centering 
	\includegraphics[width=\textwidth]{figs/top-level.png}
	\caption{Top-level assurance pattern structure.}
	\label{fig:top-level} 
\end{figure}


% Describe at a high-level how each workflow step should be assured.  Refer to figure.  Then dive into BriefCASE assurance pattern.

% Security requirements are correct and complete
\subsection{Cyber Requirement Correctness and Completeness}
BriefCASE currently includes two cybersecurity plug-ins, GearCASE~\cite{gearcase2020} and DCRYPPS~\cite{dcrypps2019}, that analyze an AADL model and output a set of cyber requirements corresponding to vulnerabilities detected in the model.  
BriefCASE maintains the generated requirements within the framework as assurance goals using the Resolute tool~\cite{resolute2014}.  
In addition to providing an AADL annex grammar for representing assurance cases, Resolute includes an evaluation engine for determining whether sufficient evidence exists (both internal and external to the AADL workspace) to support assurance claims.  
Because BriefCASE manages the development artifacts associated with the CASE workflow, it automatically provides Resolute with instructions on how those artifacts can be used to support specific cyber-resiliency goals.
%Resolute generates assurance arguments in a tree format, but also supports export to graphical tools such as AdvoCATE~\cite{advocate}.  


The assurance argument for cybersecurity requirement correctness and completeness is shown in Figure~\ref{fig:req-correct-complete}.  
%
In the figure, it can be seen that in order to support the requirement correctness and completeness claim, we must provide evidence that the full set of cyber requirements passed through a review process, were imported into the BriefCASE environment as Resolute goals or omitted with rationale, and that successive analyses on updated versions of the model finds no new vulnerabilities.  The latter reflects the iterative step in the workflow (depicted by the left-pointing arrow in Figure~\ref{fig:workflow}), in which a modified model must be re-analyzed after applying a mitigation for a previously generated requirement.  This is necessary in order to demonstrate that the mitigation of one vulnerability does not inadvertently introduce other vulnerabilities.  To argue that the current model was analyzed appropriately, we must be able to demonstrate that the model is well-formed; that is, it complies with modeling guidelines, that the analysis was indeed performed on the current version of the model, and that the analysis does not produce any new applicable requirements.

\begin{figure}[h] 
	\centering 
	\includegraphics[width=\textwidth]{figs/req-correct-complete.png}
	\caption{Assurance pattern for security requirement correctness and completeness.}
	\label{fig:req-correct-complete} 
\end{figure}

% Model assurance
\subsection{Cyber Requirements are Satisfied in the System Model}
BriefCASE includes a library of automated model transformations corresponding to common cyber requirement classes.  Each transformation modifies the model to harden it against a specific vulnerability, thereby mitigating the associated threat and addressing the driving requirement.  In addition, the transformations automatically update the corresponding Resolute goals with instructions that enable Resolute to evaluate whether the goal is supported by the necessary evidence.

% NOT SURE HOW MUCH SPACE WE WILL HAVE HERE.
% USE FILTER AS EXAMPLE
% OTHERWISE, JUST REFERENCE DESTION PAPER 
Because the transformations modify the architecture model in different ways, assuring that a specific requirement is satisfied in the model will be argued according to a transformation-specific pattern.  For example, a well-formed message requirement on a component's input port can be addressed by inserting a filter on the communication channel upstream of the target component.  The corresponding assurance pattern for this mitigation is shown in Figure~\ref{fig:filter}.  Here, we argue that the well-formed message requirement has been satisfied in the model by showing that formal verification passed on the current version of the model, that a filter is appropriately inserted upstream of the target component, and that the filter cannot be bypassed.  Assurance patterns corresponding to all the BriefCASE model transformations have been defined and are packaged with the tool.  Due to space limitations we do not include them here, but instead refer the reader to the BriefCASE project site.

\begin{figure}[h] 
	\centering 
	\includegraphics[width=\textwidth]{figs/filter.png}
	\caption{Pattern for assuring proper filter mitigation in the architecture model.}
	\label{fig:filter} 
\end{figure}

% Implementation assurance
\subsection{Cyber Requirements are Satisfied in the Realization of the System Model}
% Where BriefCASE implementations can come from (legacy/manual implementation, SPLAT synthesis, pre-packaged - Attestation, seL4, HAMR)

In the CASE workflow, a software component implementation could have various origins.  It could be legacy, third-party, or manually implemented code. It could also be generated from a behavioral model such as Simulink or be synthesized directly from the component's contract.  In BriefCASE, the latter is performed by the SPLAT tool~\cite{case-verified-filter}, which synthesizes the implementation of high-assurance components in the CakeML language~\cite{cakeml} directly from a formal assume-guarantee contract specified in the AADL component's AGREE annex~\cite{compositional-analysis-agree}.  In addition to providing a formally verified compiler, CakeML enables SPLAT to generate a proof that the synthesized implementation is correct with respect to the formal contract.

Application infrastructure code and the operating system itself must also be implemented and integrated into a deployable system.  BriefCASE employs the HAMR build tool~\cite{hamr}, which generates the infrastructure code, along with correspondence proofs that the inter-component connections specified in the model are maintained in the implementation and that no new connections have been created.  For high-assurance systems, this is made possible in part by building to a target platform running seL4~\cite{sel4-cacm18}, a formally verified microkernel that provides time and space partitioning guarantees.

% Add a paragraph (and maybe a figure?) on the HAMR API for component code to tie into following sentence
TODO - BRIEF DESCRIPTION OF HAMR API

\begin{figure}[h] 
	\centering 
	\includegraphics[width=\textwidth]{figs/deployment-interface.png}
	\caption{(a) Example AADL component interface features and disclosed component state. (b) Schematic of deployed component, with notion of observation points.}
	\label{fig:deployment-interface} 
\end{figure}

The structure of the \textit{model realization} branch of the assurance pattern (partially shown in Figure~\ref{fig:req-satisfied-in-model-realization}) therefore necessarily focuses on evidence of correctness in terms of behaviors observed at deployment observation points associated with the BriefCASE workflow.  
%Patterns for assuring deployed code with provenance external to BriefCASE are not presented here, but would be included in a straight-forward manner.


\begin{figure}[h] 
	\centering 
	\includegraphics[width=\textwidth]{figs/req-satisfied-in-model-realization.png}
	\caption{Assurance pattern for arguing the requirement is satisfied in the realization of the model.}
	\label{fig:req-satisfied-in-model-realization} 
\end{figure}

To support the claim that the deployed software components satisfy their requirements,
%in Figure~\ref{fig:req-satisfied-in-model-realization}
for each component, we require evidence that the cyber requirements are stated in terms of a component's AADL interface and publicly disclosed state, and that the component application code conforms to both its declared interface and requirements.  Although demonstrating that requirements are stated in terms of a component's interface is not strictly necessary in the general case, it is included in this pattern for confidence that the CASE workflow was followed correctly. 


Furthermore, the component's application code must conform to its declared interface and requirements.  This goal is substantiated by the argument in Figure~\ref{fig:code-conforms-to-interface-and-requirements}.
%
% The component's AADL runtime infrastructure code satisfies AADL port and threading semantics
%
Finally, the component's platform deployment context achieves its required assurance properties.  This is supported by the argument in Figure~\ref{fig:platform-deployment-context-achieves-assurance-properties}.


\begin{figure}[h]
	\centering 
	\includegraphics[width=\textwidth]{figs/code-conforms-to-interface-and-requirements.png}
	\caption{Assurance pattern for arguing component code conforms to the specified interface and requirements.}
	\label{fig:code-conforms-to-interface-and-requirements} 
\end{figure}


\begin{figure}[h] 
	\centering 
	\includegraphics[width=\textwidth]{figs/platform-deployment-context-achieves-assurance-properties.png}
	\caption{Assurance pattern for arguing platform deployment context achieves the required assurance properties.}
	\label{fig:platform-deployment-context-achieves-assurance-properties} 
\end{figure}



In addition to demonstrating that \textit{deployed} software components satisfy their requirements, we must also argue that the platform components guarantee their required properties and the system implementation preserves them.

% Show how pattern is used in BriefCASE
% Include pattern library (aadl contribution)
% Auto insert of top-level goal
% Instantiation
% Evaluation of external artifacts
\section{Pattern Instantiation and Evaluation in BriefCASE}
\label{sec:evaluation}

% Specifying evidence external to the model

\section{Conclusion}
\label{sec:conclusion}

In this paper, we have presented a collection of hierarchical cyber-resiliency assurance patterns, which are bundled with our BriefCASE framework and instantiated with a specific system under development.  Automated instantiation and evaluation of the patterns provides us with confidence that we have adequately analyzed the system for cyber vulnerabilities and addressed the corresponding cyber requirements in the system design \textit{and} implementation.
Although these patterns mainly correspond to the automated BriefCASE features that support the CASE workflow outlined in Fig.~\ref{fig:workflow}, BriefCASE is not necessarily required to use them; with minor alteration they can be applied to any tool chain that adheres to the workflow.

Unfortunately, it will rarely be the case that an entire high-assurance system can be developed in this fashion.  Not all cyber requirements can be mitigated by an automated model transformation.  Not all component implementations can be synthesized in a provably correct manner.  And not all evidential development artifacts can be automatically evaluated.
%
Although these assurance patterns provide confidence that (a) the CASE workflow was properly followed for a specific system development configuration and (b) the resulting deployable system is acceptably cyber-resilient, additional patterns are still necessary to support typical development efforts we see in practice today.  
%
Our cyber-resiliency patterns are structured hierarchically, enabling straight forward insertion of additional pattern fragments corresponding to new cyber vulnerability mitigations, processes, and workflows. 
We anticipate working towards filling these gaps on future research projects as well as with contributions from the wider security assurance community.

\section{Acknowledgment}

This work was funded by DARPA contract HR00111890001. The
views, opinions and/or findings expressed are those of the authors
and should not be interpreted as representing the official views or
policies of the Department of Defense or the U.S. Government.

\bibliographystyle{IEEEtran}
\bibliography{biblio}

\end{document}
