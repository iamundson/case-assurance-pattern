In previous work~\cite{resolute-destion}, we described assurance patterns corresponding to specific BriefCASE cyber-vulnerability mitigations.  However, we only provided a high-level summary of those patterns, and they primarily focused on assuring that security requirements were satisfied in the design by examining the model structure (for example, demonstrating that an inserted filter component could not be bypassed).
%However, those patterns primarily focused on assuring that aspects of security requirements were satisfied in the design by examining the model structure (for example, demonstrating that an inserted filter component could not be bypassed).  
%Moreover,~\cite{resolute-destion} is a short-format paper that only gives a very high-level summary of the selected patterns.
%
% CONTRIBUTION AND PAPER OUTLINE
In this paper we significantly expand on that work by presenting a more comprehensive collection of hierarchical CASE assurance patterns covering the generation and ingestion of cyber requirements, requirement satisfaction in the model, and requirement satisfaction in the realization of the model.  In addition, we describe the mechanisms by which evidence generated as part of the BriefCASE workflow is incorporated into an instantiated assurance case.

The high-level structure of our CASE assurance patterns is inspired in part by the D-MILS argument pattern~\cite{dmils}, in which system dependability properties are assured via modules arguing component, compositional, and implementation correctness.  Similar to BriefCASE, the authors demonstrate how to instantiate the D-MILS pattern from an AADL system model; however, they accomplish this via specification of an additional \textit{weaving model}, which is not necessary in BriefCASE due to the tight coupling between the modeling environment and assurance tool.

The VERDICT~\cite{verdict} framework was also developed on the CASE program and has some similarities with BriefCASE.  Although VERDICT does generate and evaluate assurance arguments based on analyses performed as part of the tool workflow, the assurance arguments are only shallow fragments of a comprehensive cybersecurity case, and focus primarily on whether applicable Common Attack Pattern Enumerations and Classifications (CAPEC)~\cite{capec} have been addressed in the design.  In contrast, our CASE assurance patterns consider vulnerability mitigations in the system design \textit{and} the implementation, and include arguments for several other aspects of cyber-resiliency assurance as well.


The Architecture-driven Multi-concern and Seamless Assurance and Certification of Cyber-Physical Systems (AMASS)~\cite{AMASS} framework incorporates a workflow similar to BriefCASE, but has a scope that encompasses assurance for all types of dependability properties, and a more ambitious goal of driving down certification costs for high-assurance systems development. AMASS supports tools and processes for system design, verification and validation activities and assurance generation, among others.  It is extensible and has a growing ecosystem supported by a European consortium of researchers and practitioners.  In contrast, BriefCASE is targeted at the development of provably cyber-resilient embedded systems, and although it does support analysis and assurance of other classes of dependability properties, the assurance patterns presented in this paper focus strictly on providing confidence of the security of the system under development.



%\todo{Describe other cybersecurity assurance patterns (NIST, etc)}
