% BriefCASE tool overview
\todo{Rephrase and clean up - this is mostly copied from IEEE Privacy \& Security paper}

BriefCASE is predicated on a Model-Based Systems Engineering (MBSE) process, in which models are the primary vehicle for communication and understanding among the parties tasked with designing the system. 
It is implemented as a collection of plugins in the Eclipse-based Open Source AADL Tool Environment (OSATE), the reference AADL modeling tool. The tools that comprise BriefCASE, as well as the mechanism for assurance generation are described below.  Additional information on BriefCASE can be found in~\cite{case-at-scale}.

\subsection{BriefCASE Architecture}

The BriefCASE architecture (see Figure~\ref{fig:briefcase-architecture}) starts with the development of a baseline AADL model of the system. BriefCASE is implemented as a collection of plugins in the Eclipse-based Open Source AADL Tool Environment (OSATE), the reference AADL modeling tool. 

\begin{figure}[h] 
	\centering 
	\includegraphics[width=\textwidth]{figs/briefcase-architecture.png}
	\caption{BriefCASE architecture.}
	\label{fig:briefcase-architecture} 
\end{figure}


%	Cyber analysis / Req generation
BriefCASE provides access to two architecture analysis tools (GearCASE~\cite{gearcase2020} and DCRYPPS~\cite{dcrypps2019}) that analyze AADL models for potential cyber vulnerabilities and generate cyber requirements for mitigation. 
Systems engineers are presented with a requirements management interface for viewing the generated requirements and importing them into the model so they can be addressed.  
Some requirements can also be formalized as assume-guarantee contracts added to the AADL model, enabling formal verification. Such a requirement will be imported into the model with with an associated formal contract.

%	Model transformation
To address the new cyber requirement, the architecture will need to be transformed in such a way as to harden the design against the vulnerability. BriefCASE provides an extendable library of model transformations for addressing common cyber vulnerabilities. 
The transformations are automated by the BriefCASE tool, resulting in a hardened model that is correct-by-construction. 
For example, the requirement that a component only receives well-formed messages can be satisfied by the insertion of a high-assurance filter. A BriefCASE transform wizard helps to configure the filter component properties, including the filter behavioral specification, which is represented as an assume-guarantee contract. BriefCASE then inserts a new filter component into the model, sets the component properties, and establishes the appropriate connections to source and destination components. The filter behavioral contract is also added to the model, enabling formal analysis of the model as well as providing the behavioral specification for a provably correct synthesis of the filter component implementation. 
The transformation also updates the assurance case with new evidential statements indicating how the associated goal has been satisfied, including the strategy used and links to context and associated evidence needed for assurance case evaluation.

%	Compositional analysis
The Assume Guarantee Reasoning Environment (AGREE) is a compositional, assume-guarantee-style model checker for AADL models~\cite{compositional-analysis-agree}. AGREE attempts to prove properties about one layer of the architecture using properties allocated to subcomponents. The composition is performed in terms of assumptions and guarantees that are provided for each component.  
Once the system architecture has been modeled in AADL and component assume-guarantee contracts have been specified, the AGREE model checker is used to verify the consistency of these contracts.
%
%1) Component interfaces – The output guarantees of each component must be strong enough to satisfy the input assumptions of downstream components.
%
%2) Correctness of implementations – The input assumptions of a system along with the output guarantees of its sub-components must be strong enough to satisfy its output guarantees. 
% 
This hierarchical strategy for reasoning about contracts, or compositional analysis, reduces the computational complexity of model checking by breaking down the larger problem into more manageable fragments.  AGREE results are automatically incorporated as evidence into the BriefCASE assurance case.

%	High-Assurance Component synthesis
The correctness of the high-assurance components inserted by BriefCASE transformations means that each such component must meet its AGREE contract. This obligation is addressed by formal synthesis, using the Semantic Properties of Language and Automata Theory (SPLAT) tool~\cite{case-models-2021}. SPLAT generates code to implement the AGREE contract and then proves that its implementation exactly preserves the meaning of the contract all the way down to the binary for the target platform.

SPLAT uses the HOL4 theorem proving system to implement the synthesis and prove its correctness relative to the contract. The synthesis targets a dialect of Standard ML called CakeML and uses CakeML’s fully verified compiler to render the final binary~\cite{cakeml}. 
The SPLAT proof shows equivalence between the contract and the synthesized CakeML, leveraging the existing CakeML compiler proof, and reasons about the perpetual re-execution of the code as scheduled in a real-time environment.

%	Infrastructure Code Generation
BriefCASE employs the High Assurance Modeling and Rapid engineering for embedded systems (HAMR) tool~\cite{hamr}, a multi-platform, multi-language AADL code generation framework. 
Using seL4 as a foundation, HAMR enables AADL to be used as a model-based development and systems engineering framework for seL4-based applications. 
%
The seL4 microkernel~\cite{sel4-sosp09} is a lightweight, fast, and secure operating system kernel. Its implementation is fully formally verified, from high-level security properties down to the binary level.
%
One of the primary objectives of HAMR is to support system builds that leverage seL4 separation and information flow guarantees to achieve the AADL-specified component isolation and inter-component communication needed for cyber-resiliency. 

For each AADL thread component, HAMR generates a thread code skeleton and application programming interfaces (APIs) for communicating over the ports declared on the component. For components that are implemented manually, the developer fills out the thread skeleton with application code. 
%
HAMR generates component infrastructure and integration code implementing the semantics of AADL-compliant thread scheduling, thread dispatching, and port-based communication. 

The seL4 deployment uses the Component Architecture for microkernel-based Embedded Systems (CAmkES) code-generation framework to configure the microkernel. The HAMR-generated CAmkES directly encodes the AADL model’s component and communication topology and includes the AADL run-time infrastructure with its thread scheduling. HAMR leverages the existing seL4 domain scheduler to enforce time partitioning and provide static cyclic scheduling. 
%
As part of its code generation process, HAMR produces flow graphs reflecting the inter-component information flow at both the AADL architecture level and the CAmkES level for the seL4 deployment. Visual representations are provided for manual inspection, and SMT-based representations are generated for formal reasoning. The SMT-based representations are used to prove that 1) all AADL modeled flows are in the CAmkES configuration, and 2) no extraneous flows have been added to the CAmkES configuration. 


\subsection{Assuring Systems Developed in BriefCASE}

% Resolute
Each of the BriefCASE tools contribute to an aspect of high-assurance systems development, and each emit evidence of correctness that can be used to substantiate cyber-resiliency assurance goals. The Resolute tool~\cite{resolute2014} is used to evaluate this evidence and incorporate it into a system cyber-resiliency assurance case. 
%
%	Very high-level overview of Resolute
Resolute is both a tool and language for embedding an assurance argument in a system architecture model and evaluating the validity of the associated evidence.

%	Maintaining cyber requirements in Resolute
A BriefCASE project contains a repository for requirements. Imported requirements (e.g., those generated by GearCASE or DCRYPPS) are represented as assurance case goals to be satisfied. 
For example, a requirement that specifies a target component shall only receive well-formed messages is imported as the Resolute \texttt{goal} depicted in Figure~\ref{fig:resolute-requirement}a.  The goal is initially marked \texttt{undeveloped}.


\begin{figure}[h] 
	\centering 
	\includegraphics[width=\textwidth]{figs/resolute-requirement.png}
	\caption{(a) Cyber requirement imported as \textit{undeveloped} Resolute assurance goal.  (b) Updated goal with logical rules for determining whether goal is satisfied.}
	\label{fig:resolute-requirement} 
\end{figure}

%	Updating requirements with automated transform assurance library in BriefCASE
The well-formed message requirement can be mitigated by performing an automated model transformation for inserting a filter. Each transformation has an associated assurance pattern that describes a strategy for determining from the model whether the requirement has been satisfied (see Section~\ref{sec:requirements-satisfied-in-model}).  
These evidential statements are added to the goal as the design is updated to address this requirement, as shown in Figure~\ref{fig:resolute-requirement}b.  For the insertion of a filter, Resolute must now check that AGREE formal analysis passes and the filter was added correctly to the model.
Subsequent changes to the model that invalidate any of the assurance claims can then be detected and corrected.  

%	Specifying external evidence in Resolute
Resolute has recently been updated to enable evaluation of artifacts external to the modeling workspace. An Artifact Management tool is included with BriefCASE for specifying how Resolute should parse documents with specific formats (such as test results, review forms, etc.) to determine whether they support specific assurance claims.  For each type of document, users can specify a regular expression that gets matched against the contents of the document, such that a correct match indicates the validity of the evidence in supporting a specific claim.

%	BriefCASE assurance pattern in Resolute
The assurance patterns presented in our previous work (see~\cite{resolute-destion}) primarily focused on design assurance; that is, correctness of the architecture model.  With recent updates to BriefCASE and Resolute, we have since expanded the pattern into a comprehensive cyber-resiliency assurance argument covering architecture design, code generation, and system build.  The assurance pattern is formalized in Resolute and included with BriefCASE.
%
We describe the BriefCASE system cyber-resiliency assurance pattern in the next section.
